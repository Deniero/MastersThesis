\chapter{Introduction}
Automated driving has made considerable advancements in the last 15 years, yet significant technical challenges still persist (\cite{ayoub_manual_2019}).
Fully developed and integrated, it will have a lasting impact on mobility, road safety as well as society in general (\cite{milakis_policy_2017}).

Different complexity levels of driving automation are defined by the Society of Automotive Engineers (SAE) \todo{cite J3016\_202104} with a level 5 system being required to perform the entire task of diving completely without a human driver. Lower levels take over only parts of the driving, such as parking, steering, etc ... Examples are driver support features such as Automatic Emergency Braking (AEB) or Automatic Lane Keep Systems (ALKS).

Automated driving systems are considered safety-critical and need responsible and throughout testing in order to minimize traffic accidents and human harm. According to \cite{kluck_search-based_2022}, the main requirements of automated driving systems are that they "must operate under all circumstances" and, that these systems "must be safe". Self-driving systems have the potential for harm, as their failure may lead to catastrophic accidents. Due to the high complexity of the environment, in which automated driving systems are applied, high quality testing is needed. 

\begin{quote}
	\begin{em}
		\enquote{Thoroughly validating and verifying automated or autonomous driving functions is inevitable for assuring to meet quality criteria for safety-critical systems.} (\cite{felbinger_comparing_2019})
	\end{em}
\end{quote}

According to \cite{kalra_driving_2016}, a huge amount of miles driven by an autonomous driving system is required to demonstrate and test their statistical safety compared to human driving. Driving this amount might not be feasible in the real world.
\cite{kluck_search-based_2022} adds, that unguided mileage collection may provide only "limited quality assurance". During real world driving, the number of challenging driving situations is low, as most everyday situations can be driven without trouble.

According to \cite{kluck_search-based_2022}, search based testing (SBT) is a possible tool for improving the robustness and revealing systematic failures of automated driving systems. SBT generates inputs for a system under test in order to trigger some unwanted behaviour. It utilizes a problem-specific cost function that guides the search towards areas that have a high probability of failures. SBT can be applied to complex and large search spaces, where other, less focused, testing tools like full factorial or random testing might struggle.

Genetic algorithms (GAs) are a viable tool of search based testing and were already successfully utilized in the domain of automated driving, as evidenced by notable studies such as \cite{klampfl_using_nodate}, \cite{felbinger_comparing_2019}, \cite{kaufmann_critical_2021} and \cite{almanee_scenorita_2021}. 

This thesis will use a genetic algorithm in order to generate critical driving scenarios for testing self-driving technology in vehicles. While generating these scenarios is the objective, the main task of the thesis will evolve around the implementation of the genetic algorithm as well as the optimization of its hyperparameters.

\section{Research Questions}
The following two research questions will provide a guideline for this master's thesis. Both questions evolve around examining and comparing the performance of a genetic algorithm on the given search space which is explained in chapter \ref{chap:implementation}.

\subsection{Research Question 1}
\begin{quote}
	\begin{em}
		\textit{Is a genetic algorithm suitable for generating critical driving scenarios compared to random search?}
	\end{em}
\end{quote}

Although the previously presented papers provide strong suggestions that a genetic algorithm has a significant advantages over random search, a comparison is still highly important. Are situations possible, where a genetic algorithm struggles to find critical scenarios compared to a random approach?


\subsection{Research Question 2}
\begin{quote}
	\begin{em}
		\textit{Is it possible to improve the performance of a genetic algorithm by optimizing the control parameter using the taguchi method?}
	\end{em}
\end{quote}

In order tune the hyperparameter of the genetic algorithm, design of experiment, specifically the taguchi method was used. In an effort to test, if significant performance improvements are possible, the optimized GA will be compared to a GA using control parameters as recommended by existing literature. This evaluation aims to understand, whether notable performance advancements for genetic algorithms are possible by deploying this method.























