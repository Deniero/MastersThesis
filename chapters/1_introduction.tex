\chapter{Introduction}
This Thesis will use a Genetic Algorithm in order to generate critical Driving Scenarios for testing ADAS/AD Functionality in vehicles.
While generating these scenarios is the objective, the main task of the thesis will evolve around the implementation of the Genetic Algorithm as well as the Optimization of its Hyperparameter.


\section{Research Questions}
\subsection{Research Question 1}
\textit{Is a Genetic Algorithm suitable for generating critical driving scenarios compared to a random generation?}

\subsection{Research Question 2}
\textit{Can hypertuning improve the performance of a Genetic Algorithm?}

\subsection{Research Question 3}
\textit{Can a hypertuned Genetic Algorithm generalize on different start scenarios?}


\section{Shortcomings}
This Master Thesis started with the developement of the Traffic Manger and thus progress was closely linked. Without a working simulations, no genetic alogirthmis could be tested. Due to time and performance constraints, it is not possible to test a full driving stack like autoware, as well as other professional ADAS/AD functions.
In this Thesis, internal functions like Time-To-Collision and Emergency Braking will be optimized. The learned  information on e.g. optimal hyperparameter settings can then be applied in further steps to test these functions. This will however not be tackled by this thesis.

Performance is also a problem and will lead to many shortcuts that need to be taken. There is a hughe number of possible compations of hyperparamter, so only a handful can be tested. In further chapers, these shortcuts will be explained and their relevancy will be dicussed.