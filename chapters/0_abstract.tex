%%%% Time-stamp: <2013-02-25 10:31:01 vk>
\chapter*{Abstract}
\label{chap:abstract}

With the advent of autonomous driving, ensuring the correct implementation and safety of these systems becomes crucial. Relying solely on real-world driving for testing might not be able to accumulate enough mileage; thus testing autonomous driving functions will need to be supported using simulations and software. This master's thesis will deploy a search-based approach for generating critical traffic situations in a simulation environment. Utilizing the dispersed search power of genetic algorithms, the vehicles in the simulation are controlled in a way, that will generate critical traffic situations for the autonomous driving function. For the purposes of this master's thesis, the tested vehicle lacks complete autonomous driving capabilities. Instead, a behavior tree guides it through the simulated world.

After establishing the used tools and discussing their implementation, this master's thesis will focus on optimizing the control parameters of the genetic algorithm. Design of experiment, namely the Taguchi method was applied for that purpose. This optimized algorithm will be subsequently compared in different start scenarios to a genetic algorithm with control parameters taken from existing literature as well as with random search. The results of this evaluation show, that utilizing the Taguchi method is a very effective way of tuning a genetic algorithm on the given search problem. The optimized genetic algorithm displayed significantly better performance compared to random search and to a genetic algorithm which utilizes hyperparameters suggested by existing literature. Only in the case of a small number of actors, the performance between both genetic algorithms was similar.


\chapter*{Kurzfassung}
\label{chap:kurzfassung}
Mit dem Aufkommen des autonomen Fahrens wird eine Gewährleistung der korrekten Implementierung und Sicherheit dieser Systeme entscheidend. Sich allein auf das Testen durch Fahren in der realen Welt zu verlassen, reicht möglicherweise nicht aus. Daher müssen Tests für autonome Fahrfunktionen durch Simulationen und Software unterstützt werden. Diese Masterarbeit wendet einen suchbasierten Ansatz an, um mit Hilfe einer Simulationsumgebung, kritische Verkehrssituationen zu generieren. Dies wird durch die Nutzung der breitflächigen Suche eines Genetischen Algorithmus erreicht. Der Genetische Algorithmus wird Fahrzeuge in der Simulation steuern und somit kritische Situationen für die autonome Fahrfunktion erzeugen. Das getestet Fahrzeug verfügt über keine vollständige autonome Fahrfähigkeit. Stattdessen wird ein ‘Behavior Tree‘ (zu Deutsch: Verhaltensbaum) verwendet, um es durch die Welt zu führen.

Nachdem die verwendeten Werkzeuge etabliert und ihre Implementierung besprochen wurde, wird sich diese Masterarbeit auf die Optimierung der Konfigurationsvariablen des Genetischen Algorithmus konzentrieren. Zu diesem Zweck wird eine statistische Versuchsplanung, insbesondere die Taguchi Methode, angewendet. Der optimierte Algorithmus wird anschließen in verschiedenen Startszenarien mit einem Genetischen Algorithmus, welcher Konfigurationsvariablen aus bestehender Literatur verwendet, als auch mit Zufallssuche verglichen. Die Ergebnisse dieser Bewertung zeigen, dass die Anwendung der Taguchi Methode eine sehr effektive Möglichkeit ist, einen genetischen Algorithmus auf das gegebene Suchproblem abzustimmen. Der optimierte genetische Algorithmus zeigte im Vergleich zur zufälligen Suche und zu einem genetischen Algorithmus, der Hyperparameter aus der vorhandenen Literatur verwendet, signifikant bessere Leistungen. Nur in Szenarien mit geringer Anzahl an Fahrzeugen war die Leistung zwischen beiden genetischen Algorithmen vergleichbar.

\chapter*{Acknowledgements}
\label{chap:acknowledgements}
I would like to express my deep gratitude to my supervisor Univ.Prof. Dipl.-Ing. Dr.techn. Franz Wotawa for his guidance, feedback and support.

This master's thesis was a company collaboration with AVL List GmbH in Graz, so I thank my colleagues Florian Klück, Lorenz Klampfl, David Kaufmann and Gabriel Muresan for their shared expertise as well as numerous constructive discussions.

Finally, I thank my parents Petra and Ernst, my sister Christina and my partner Johanna for their support and patience throughout the entire process of completing this master's thesis.



%\glsresetall %% all glossary entries should be used in long form (again)
%% vim:foldmethod=expr
%% vim:fde=getline(v\:lnum)=~'^%%%%\ .\\+'?'>1'\:'='
%%% Local Variables:
%%% mode: latex
%%% mode: auto-fill
%%% mode: flyspell
%%% eval: (ispell-change-dictionary "en_US")
%%% TeX-master: "main"
%%% End:
